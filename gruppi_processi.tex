\section{Gruppi di processi}
\subsection{Gestione processi in Unix}
All'interno di Unix i processi vengono raggruppati secondi vari criteri, dando vita a sessioni, gruppi e threads.
I process groups consentono una migliore gestione dei segnali e della comunicazione tra i processi. Un processo, per l'appunto, può:
\begin{itemize}
    \item Aspettare  che  tutti  i  processi  figli  appartenenti  ad  un determinato  gruppo terminino;
    \item Mandare un segnale a tutti i processi appartenenti ad un determinato gruppo.
\end{itemize}
Esempio di utilizzo dei gruppi:
\begin{lstlisting}[language=C]
    waitpid(-33,NULL,0); // Wait for a children in group 33
    kill(-33,SIGTERM); // Send SIGTERM to all children in group 33
\end{lstlisting}
Mentre, generalmente, una sessione è collegata ad un terminale, i processi vengono raggruppati nel seguente modo:
\begin{itemize}
    \item In bash, processi concatenati tramite pipes appartengono allo stesso gruppo: \begin{verbatim}
    cat /tmp/ciao.txt | wc -l | grep '2'
    \end{verbatim}
    \item Alla loro creazione, i figli di un processo ereditano il gruppo del padre
    \item Inizialmente, tutti i processi appartengono al gruppo di 'init', ed ogni processo può cambiare il suo gruppo in qualunque momento.
\end{itemize} 
Il processo il cui PID è uguale al proprio GID è detto process group leader.

\subsection{Group System Calls}
\texttt{int setpgid(pid$\_$t pid, pid$\_$t pgid); //set GID of proc. (0=self)} \href{https://www.man7.org/linux/man-pages/man3/setpgid.3p.html}{MANPAGE}\\
La funzione setpgid() deve unirsi a un processo esistente
o creare un nuovo gruppo di processi all'interno della sessione di
processo di chiamata.

L'ID del gruppo di processi di un leader di sessione non deve cambiare.

In caso di completamento con esito positivo, l'ID del gruppo di processi che corrisponde a pid deve essere impostato su pgid.

Dopo il completamento con successo, setpgid() restituirà 0; altrimenti,
-1 e errno deve essere impostato per indicare l'errore.\\
       
\texttt{pid$\_$t getpgid(pid$\_$t pid); // get GID of process (0=self)} \href{https://www.man7.org/linux/man-pages/man3/getpgid.3p.html}{MANPAGE}\\
La funzione getpgid() restituirà l'ID del gruppo di processi del file
processo il cui ID processo è uguale a pid. Se pid è uguale a 0,
getpgid() restituirà l'ID del gruppo di processi della chiamata
processi.

Dopo il completamento con successo, getpgid() restituirà il process ID del gruppo. Altrimenti, restituirà (pid$\_$t) -1 e imposterà errno per indicare l'errore. \\
 \paragraph{Esempio 30:}\hfill \break
\lstinputlisting[language=C]{Cap3/esempio30.c}

\paragraph{Esempio 31, Mandare segnali ai gruppi:}\hfill \break
\lstinputlisting[language=C]{Cap3/esempio31.c}

\begin{enumerate}
    \item Processo 'ancestor' crea un figlioa.Il figlio cambia il proprio gruppo e genera 3 figli (Gruppo1)b.I 4 processi aspettano fino all'arrivo di un segnale
    \item Processo 'ancestor' crea un secondo figlioa.Il figlio cambia il proprio gruppo e genera 3 figli (Gruppo2)b.I 4 processi aspettano fino all'arrivo di un segnale
    \item Processo 'ancestor' manda due segnali diversi ai due gruppi
\end{enumerate}


\paragraph{Esempio 32, Wait figli in un gruppo:}\hfill \break
\lstinputlisting[language=C]{Cap3/esempio32.c}

\href{https://linux.die.net/man/2/setpgid}{more info}
